\documentclass[a4paper,10pt]{article}

\usepackage[utf8]{inputenc}
\usepackage[spanish]{babel}
\usepackage{graphicx}
\usepackage{xcolor}
\usepackage{geometry}
\usepackage{multicol} 
\usepackage{lipsum}   
\usepackage{booktabs} 
\usepackage{cite}     
\usepackage{titlesec} % Para personalizar títulos

% --- CONFIGURACIÓN ESTILO IEEE ---
% 1. Redefinir numeración: Romano (sección), Letra (sub), Arábigo (subsub)
\renewcommand{\thesection}{\Roman{section}}
\renewcommand{\thesubsection}{\Alph{subsection}}
\renewcommand{\thesubsubsection}{\arabic{subsubsection}}

% 2. Formato de los títulos
% Sección: Centrada, Versalitas (Small Caps) y con punto
\titleformat{\section}[block]{\bfseries\scshape\centering}{\thesection.}{1em}{}

% Subsección: Cursiva, alineada a la izquierda
\titleformat{\subsection}[block]{\itshape}{\thesubsection.}{1em}{}

% Sub-subsección: Cursiva, con sangría y paréntesis
\titleformat{\subsubsection}[runin]{\itshape}{\thesubsubsection)}{1em}{}[.---]
% ---------------------------------

\geometry{margin=2cm, top=1.5cm}
\definecolor{uisgreen}{RGB}{120,160,60}

\begin{document}

% ===== ENCABEZADO ALINEADO =====
\noindent
\begin{minipage}[c]{0.5\textwidth}
    \includegraphics[height=1.2cm]{images/uis_logo.png} 
    \begin{minipage}[c]{0.7\textwidth}
    \vspace{-22pt}
    \color{uisgreen}\scriptsize\sffamily
    \textbf{Universidad Industrial de Santander}\\
    Escuela de Ingenierías Eléctrica,\\
    Electrónica y Telecomunicaciones
\end{minipage}
\end{minipage}
\hfill
\begin{minipage}[c]{0.48\textwidth}
    \raggedleft
    \begin{minipage}[c]{0.75\textwidth}
        \color{uisgreen}\scriptsize\sffamily
        Laboratorio de \textbf{COMUNICACIONES II (27145)}\\
        Práctica 1: \textbf{PROGRAMACIÓN EN RADIO}\\
        \textbf{DEFINIDA POR SOFTWARE (GNURADIO)}\\
        Grupo: \textbf{A1 - G3}
    \end{minipage}
    \raisebox{-8pt}{\includegraphics[height=1.2cm]{images/e3t_logo.png}}
\end{minipage}

\vspace{0.1cm}
{\color{uisgreen}\rule{\textwidth}{2pt}}
\vspace{0.5cm}

% ===== TÍTULO PRINCIPAL =====
\begin{center}
    \Large\textbf{Práctica 1: Programación en Radio Definida por Software (GNU Radio)}
\end{center}

\vspace{0.4cm}

\begin{multicols}{2}

\section{Metodología}

\section{Implementación del acumulador}

\section{Implementación del diferenciador}

\section{Implementación del bloque estadístico}

\section{Aplicacion}


\section{Conclusiones}


\begin{thebibliography}{99}
    \bibitem{referencia1} 
    W. Stallings, \textit{Comunicaciones y Redes de Computadores}, 7ma ed. Prentice Hall, 2004.
    
    \bibitem{referencia2}
    GNU Radio Project, ``Documentation'', [En línea]. Disponible en: \texttt{https://www.gnuradio.org/}.
\end{thebibliography}

\end{multicols}

\end{document}
