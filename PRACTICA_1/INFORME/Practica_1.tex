\documentclass[a4paper,10pt]{article}

\usepackage[utf8]{inputenc}
\usepackage[spanish]{babel}
\usepackage{graphicx}
\usepackage{xcolor}
\usepackage{geometry}
\usepackage{multicol} 
\usepackage{lipsum}    
\usepackage{booktabs} 
\usepackage{cite}      
\usepackage{titlesec} 
\usepackage{amsmath} % Añadido para las ecuaciones matemáticas

% --- CONFIGURACIÓN ESTILO IEEE ---
\renewcommand{\thesection}{\Roman{section}}
\renewcommand{\thesubsection}{\Alph{subsection}}
\renewcommand{\thesubsubsection}{\arabic{subsubsection}}

\titleformat{\section}[block]{\bfseries\scshape\centering}{\thesection.}{1em}{}
\titleformat{\subsection}[block]{\itshape}{\thesubsection.}{1em}{}
\titleformat{\subsubsection}[runin]{\itshape}{\thesubsubsection)}{1em}{}[.---]
% ---------------------------------

\geometry{margin=2cm, top=1.5cm}
\definecolor{uisgreen}{RGB}{120,160,60}

\begin{document}

% ===== ENCABEZADO =====
\noindent
\begin{minipage}[c]{0.5\textwidth}
    % Asegúrate de que el logo esté en la carpeta 'images'
    \includegraphics[height=1.2cm]{images/uis_logo.png} 
    \begin{minipage}[c]{0.7\textwidth}
        \vspace{-22pt}
        \color{uisgreen}\scriptsize\sffamily
        \textbf{Universidad Industrial de Santander}\\
        Escuela de Ingenierías Eléctrica,\\
        Electrónica y Telecomunicaciones
    \end{minipage}
\end{minipage}
\hfill
\begin{minipage}[c]{0.48\textwidth}
    \raggedleft
    \begin{minipage}[c]{0.75\textwidth}
        \color{uisgreen}\scriptsize\sffamily
        Laboratorio de \textbf{COMUNICACIONES II (27145)}\\
        Práctica 2: \textbf{TRANSMISIÓN DIGITAL}\\
        \textbf{ESTADÍSTICAS Y PSD}\\
        Grupo: \textbf{A1 - G1}
    \end{minipage}
    \raisebox{-8pt}{\includegraphics[height=1.2cm]{images/e3t_logo.png}}
\end{minipage}

\vspace{0.1cm}
{\color{uisgreen}\rule{\textwidth}{2pt}}
\vspace{0.5cm}

\begin{center}
    \Large\textbf{Informe de Práctica 1: Programacion en radio definida por software ( GNU Radio )}
\end{center}

\vspace{0.4cm}

\begin{multicols}{2}

\section{Metodología}
En esta sección se describe el flujo de trabajo en GNU Radio. Se utilizó un generador de bits aleatorios conectado a un filtro FIR para dar forma al pulso. La tasa de bits se definió como $R_b = 32000$ bps \cite{referencia2}.

\section{Bloque Diferencial}
El bloque diferencial calcula la derivada discreta de la señal. La ecuación que rige este comportamiento es:
\begin{equation}
    y[n] = \frac{x[n] - x[n-1]}{T_s}
\end{equation}

% INSERTAR IMAGEN AQUÍ
\begin{center}
    \includegraphics[width=\linewidth]{images/bloque_diferencial.png}
    \captionof{figure}{Bloque diferencial implementado.}
\end{center}

\section{Análisis de la PSD}
Se observó la Densidad Espectral de Potencia (PSD) para diferentes valores de $Sps$. Como se menciona en \cite{referencia1}, el ancho de banda depende directamente de la forma del pulso.

\section{Conclusiones}
La implementación en GNU Radio permite validar de manera práctica los conceptos teóricos de la comunicación digital.

% IMPORTANTE: Cerramos multicols antes de la bibliografía para evitar errores de formato
\end{multicols}

\vspace{0.5cm}

\begin{thebibliography}{99}
    \bibitem{referencia1} 
    W. Stallings, \textit{Comunicaciones y Redes de Computadores}, 7ma ed. Prentice Hall, 2004.
    
    \bibitem{referencia2}
    GNU Radio Project, ``Documentation'', [En línea]. Disponible en: \texttt{https://www.gnuradio.org/}.
\end{thebibliography}

\end{document}
